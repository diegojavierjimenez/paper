% !TeX root = paper.tex
% !TEX output_directory = .textmp

\documentclass[11 pt]{article} 

% !TeX root = paper.tex

%% DOCUMENT ENCODING ===================================================
\usepackage[utf8]{inputenc}
\usepackage[T1]{fontenc}


% SILENCE WARNINGS =====================================================
\usepackage{silence}
\WarningFilter{biblatex}{Using fall-back BibTeX(8) backend:}
\WarningFilter{biblatex}{'\mkdatezeros' is deprecated.}

%% BIBLIOGRAPHY ========================================================
\usepackage{csquotes}
\usepackage{xpatch}
\usepackage{natbib}

%% PAGE FORMAT =========================================================
\usepackage{sectsty} 			
\sectionfont{\Large}
\subsectionfont{\large}

\usepackage[verbose]{geometry}
\geometry{letterpaper}
\geometry{margin=2.5cm}

\setlength{\parskip}{0.5\baselineskip}	% change line skip after a paragraph finishes

%\usepackage{parskip}

\usepackage{setspace}
\onehalfspacing
%\doublespacing

% FOOTNOTES FORMAT =====================================================
%\usepackage[bottom,hang]{footmisc}
\usepackage[marginal]{footmisc}
\setlength{\footnotesep}{0.65\baselineskip}
\setlength{\footnotemargin}{-1.5ex}


%% ADDITIONAL PACKAGES =================================================
\usepackage{graphicx}
\graphicspath{{./figures/}} 

\usepackage{array} 
\usepackage{paralist} 
\usepackage{verbatim} 
\usepackage{amsmath}
\usepackage{amsthm}
\usepackage{booktabs}
\usepackage{multirow}
\usepackage{bigstrut}
\usepackage{amssymb}
\usepackage{color}
\usepackage{bbm}
\usepackage{soul}
\usepackage{pdflscape}
\usepackage{multicol}
\usepackage{enumitem}
\usepackage[linewidth=1pt]{mdframed}
\usepackage{xspace}
\usepackage{tocloft}
\usepackage{etoc}
\usepackage{adjustbox}
\usepackage{anyfontsize}
\usepackage{ifdraft}
\usepackage{eucal}
\usepackage{rotating}
\usepackage{fancyvrb}
\usepackage{comment}
\usepackage{float}

% TODONOTES PACKAGE ====================================================
\usepackage[obeyFinal, colorinlistoftodos, textsize=small]{todonotes}
\setuptodonotes{fancyline, color=yellow, inline}

% https://mirror.hmc.edu/ctan/macros/latex/contrib/todonotes/todonotes.pdf
\newcounter{todoListItems}
\newcommand{\todoLink}[2][ ]{
% Increment counter
\addtocounter{todoListItems}{1}
\todo[%
caption={\protect\hypertarget{todo\thetodoListItems}{}Translation},
#1]
{
#2 \hfill
\hyperlink{todo\thetodoListItems}{$\uparrow$}
}
}

\newcommand{\td}{\todoLink}

\newcommand{\forAll}[2]{\td[color=yellow,inline,caption={\thetodoListItems:
    @All -- #1}]{\textbf{@All} -- #2}}
\newcommand{\forA}[2]{\td[color=green!40,inline,caption={\thetodoListItems:
    @A -- #1}]{\textbf{@A} -- #2}}
\newcommand{\forB}[2]{\td[color=blue!40,inline,caption={\thetodoListItems:
    @B -- #1}]{\textbf{@B} -- #2}}


% CAPTIONS =============================================================
\usepackage[labelfont=bf,justification=centering, position=top]{caption}
\usepackage[labelfont=bf, justification=centering, labelformat=simple]{subcaption}
\renewcommand\thesubfigure{(\alph{subfigure})}


\usepackage{pstricks}
%\usepackage[multiple]{footmisc}
\usepackage{etoolbox,xstring,xspace}

% FIXED COLUMN WIDTHS AND A WAY TO HIDE COLUMNS ========================
\newcommand{\PreserveBackslash}[1]{\let\temp=\\#1\let\\=\temp}
\newcolumntype{C}[1]{>{\PreserveBackslash\centering}p{#1}}
\newcolumntype{R}[1]{>{\PreserveBackslash\raggedleft}p{#1}}
\newcolumntype{L}[1]{>{\PreserveBackslash\raggedright}p{#1}}
\newcolumntype{H}{>{\setbox0=\hbox\bgroup}c<{\egroup}@{}}

% TIKZ PROGRAMMING =====================================================
\usepackage{tikz}
\usetikzlibrary{matrix,shapes,arrows,fit,tikzmark}
\usetikzlibrary{shapes.misc}
\usetikzlibrary{backgrounds}
\usetikzlibrary{shapes.multipart}
\usetikzlibrary{arrows.meta}
\usetikzlibrary{shapes.geometric}

%% CHANGE FONTS ========================================================
\usepackage[looser]{newpxtext} % LOOSER = BETTER WORD SPACE
\usepackage{textcomp}
\usepackage[upint]{newpxmath}
\usepackage{bm}


\definecolor{darkred}{rgb}{0.55, 0, 0}
\definecolor{ash}{RGB}{50,50,50}
\definecolor{darkred}{rgb}{0.35, 0.0, 0.0}
\definecolor{darkblue}{rgb}{0.0, 0.0, 0.5}
\definecolor{navyblue}{rgb}{0.0, 0.0, 0.8}
\definecolor{cadmiumgreen}{rgb}{0.0, 0.6, 0.24}
\definecolor{red_tol}{HTML}{D92120}
\definecolor{blue_tol}{HTML}{404096}
\definecolor{blugreen_tol}{HTML}{529DB7}
\setulcolor{darkred}

\newtheorem{theorem}{Theorem}
\newtheorem{proposition}{Proposition}
\newtheorem{lemma}{Lemma}
\newtheorem{definition}{Definition}

\newtheorem{requirement}{Requirement}
\newtheorem{assumption}{Assumption}
\newtheorem{corollary}{Corollary}

\theoremstyle{remark}
\newtheorem{example}{Example}

%\include{listings-stata}
\usepackage{xcolor}
\usepackage{tcolorbox}






%% HYPERREF SHOULD BE LAST =============================================
\usepackage{hyperref}
\hypersetup{
    pdfstartview={FitH}, % fits the width of the page to the window
    pdftitle={XXX}, % title
    pdfauthor={Diego Jimenez Hernandez}, % author
    colorlinks=true,  % false: boxed links; true: colored links
    linkcolor=darkblue, % color of internal links (change box color with linkbordercolor)
    citecolor=darkred, % color of links to bibliography
    filecolor=magenta, % color of file links
    urlcolor=darkred, % color of external links
}



% PACKAGE FOR TITLE PAGE =====================================
\usepackage{titling}
\setlength{\droptitle}{-4em}              % PUSH TITLE UPWARDS
\pretitle{\begin{center}\huge\textbf}     % MAKE TITLE HIGHE AND BOLD
\posttitle{\par\end{center}\vskip 0.5em}  % FINISH THE TITLE AND ADD SPACE
%\renewcommand{\thanksscript}[1]{}        % REMOVE THANKS MARK FROM FOOTNOTE
%\renewcommand{\makethanksmarkhook}{\fontsize{9}{11}\selectfont}

% BACKREF FOR FIGURES AND TABLES =============================
\makeatletter

\AtBeginDocument{%
\let\origref\ref
\renewcommand*\ref[1]{%
  \origref{#1}\xlabel{#1}}
}
\newrobustcmd*\xlabel[1]{%
   \ifcsdef{siteref@doc@#1}{}{\csgdef{siteref@doc@#1}{,}}%
    \@bsphack%
    \begingroup
       \csxdef{siteref@doc@#1}{\csuse{siteref@doc@#1},\thepage}%
         \protected@write\@auxout{}%
        {\string\SiteRef{siteref@#1}{\csuse{siteref@doc@#1}}}%
     \endgroup
     \@esphack%
}

\newrobustcmd*\SiteRef[2]{\csgdef{#1}{#2}}

\newrobustcmd*\xref[1]{%
\ifcsundef{siteref@#1}{%
     \@latex@warning@no@line{Label `#1' not defined}
     }{%
    \begingroup
      \StrGobbleLeft{\csuse{siteref@#1}}{2}[\@tempa]\relax%
      \def\@tempb{}%
      \@tempcnta=0\relax%
      \@tempcntb=\@ne\relax%
      \def\do##1{\advance\@tempcnta\@ne}%
      \expandafter\docsvlist\expandafter{\@tempa}%
       \def\do##1{%
         \ifnum\@tempcntb=\@tempcnta\relax%
            \hyperpage{##1}%
         \else
            \hyperpage{##1},%
          \fi%
          \advance\@tempcntb\@ne
       }%
       [\expandafter\docsvlist\expandafter{\@tempa}]\xspace%
    \endgroup
   }%
}
\makeatother

% USER-WRITTEN COMMAND DEFINITIONS ===========================
% BACKREFERENCE COMMAND
\newcommand{\backreference}[1]{%
\ifoptionfinal
    {}%
    { 

    \noindent
    {\footnotesize\textit{Backreferenced:} 
     \texttt{\detokenize{#1}} \xref{#1}}}

    \noindent%
}

% OUTLINE
\newcommand{\outline}[1]{%
\begin{tcolorbox}%
\footnotesize%
	#1
\end{tcolorbox}
}

% RED TEXT
\newcommand{\redtext}[1]{\textcolor{red}{#1}}

% XXX TO FILL OUT
\newcommand{\xx}{\textcolor{magenta}{\texttt{<XX>}}}

% TAKEAWAY COMMAND
\newcommand{\takeaway}[1]{\begin{center}\textit{#1}\end{center}}


% SINGLE SPACING IN TABLES AND FIGURES
\AtBeginEnvironment{table}{\vspace{-0.5\baselineskip}\singlespacing}
\AtBeginEnvironment{figure}{\vspace{-0.5\baselineskip}\singlespacing}

\AtEndEnvironment{table}{\vspace{-0.5\baselineskip}}
\AtEndEnvironment{figure}{\vspace{-0.5\baselineskip}}

\usepackage{titlesec}
%\titleformat{\paragraph}{\normalfont\normalsize\bfseries}{\theparagraph}{0pt}{}
\titlespacing*{\paragraph}{0pt}{1ex}{1em}
%\let\oldparagraph=\paragraph
%\renewcommand\paragraph[1]{\oldparagraph{#1.}}

%\titlespacing\section{0pt}{0.5ex}{0.5ex}
%\titlespacing\subsection{0pt}{0.5ex}{0.5ex}
%\titlespacing\subsubsection{0pt}{0.5ex}{0.5ex}


% REMOVE HBOX WARNINGS
\hfuzz=100pt








 % document preamble
\renewcommand{\P}{\mathbb{P}}
\newcommand{\as}{\overset{\text{a.s.}}{\longrightarrow}} % some numbers defined that are useful throughout the document

% PAPER TITLE =========================================================
% ALTERNATIVE TITLE:

\title{Paper Title  \\
	Paper Subtitle%
\thanks{%
Author 1 (corresponding author): Position, Institution; \url{name@mail.com}. 
Author 2: Position, Insitution. \\[1ex]
\noindent
Anne Fournier was instrumental to make this template readable. The views expressed herein are those of the authors and do not necessarily reflect the views of Employer. All errors are our own.
}}
\author{%
	\large First Author  \and
	\large Second Author
	}
\date{
	%\today \\ % If paper is online at http://paper.website.com
	date \\
	\small{[\href{http://paper.website.com}{click for latest version}]}
}



%----------------------------------------------------------------------------------------
%	DOCUMENT SETUP
%----------------------------------------------------------------------------------------
\begin{document}

\etocdepthtag.toc{main}
\etocsettagdepth{main}{subsection}
\etocsettagdepth{appendix}{none}


\maketitle
\thispagestyle{empty}
\vspace{-4ex}
\begin{abstract}

%!TEX root = ../paper.tex


\noindent
This template provides a format for academic papers. 
%
It includes example sections, subsections, equations, figures, appendices, and citations.
%
Aditionally, the template provides debugging tools used to collaborate with coauthors.

\vspace{4ex}
\noindent \textbf{JEL: }JEL Code 1; JEL Code 2; JEL Code 3; JEL Code 4. \\
\noindent \textbf{Keywords:} keyword1; keyword2; keyword3.
\end{abstract}
\newpage

\ifoptionfinal
{}%
{
\noindent
\textbf{Note:} The table of contents automatically disappears if you add \texttt{final} to the \texttt{documentclass}.
\tableofcontents
\thispagestyle{empty}
\newpage
\noindent
\textbf{Note:} The list of todos automatically disappears if you add \texttt{final} to the \texttt{documentclass}.
\listoftodos
\thispagestyle{empty}
\newpage
}




\setcounter{page}{1}


\section{Introduction}
% !TeX root = ../paper.tex


The `Paper template' folder contains this document's building blocks for an academic paper. It includes example preambles/formatting/commands and example sections, as well as a sample bibliography and appendix. This document shows how the constituent parts—section .tex files, bib files, and preamble files—can be combined into a paper format. 

Specifically, the folder contains the following files:
\begin{enumerate}
\item \Verb"preamble.tex" loads packages and defines settings.
\item  \Verb"library.bib", which contains citations.
\item \Verb"commands.tex" defines new commands. 
\item \Verb"paper.tex", and associated files that \LaTeX creates when compiling the document: an aux file, a log file, an out file, a toc file, a GZ file, and finally, the pdf output we want. This file puts together the different components of the document---compile it to get a PDF of the full paper. Also, refer to this file for settings choices like citation format (currently: Chicago author-date).
\item Folders for figures and sections.
\end{enumerate}

The first paragraph after invoking a \Verb"\section{}" command is not indented, but subsequent paragraphs are.

\subsection {Example subsection}

This is an example of a subsection. The purpose of this subsection is to use the \texttt{commands.tex} file, equations, and citations; to do so, we will use a basic math example. Consider the definition of almost sure convergence \citep{durrett2019probability}: a sequence $\left\{X_n\right\}$ converges almost surely to $X$, written $X_n \as X$, if $\P(\omega : \lim_{n \to \infty} X_n = X) = 1$.

One way to write this is:
\begin{verbatim}
$X_n \overset{\text{a.s.}}{\longrightarrow} X$ 
if 
$\mathbb{P}(\omega : \lim_{n \to \infty} X_n = X) = 1$
\end{verbatim}

In the \texttt{commands.tex} file, however, we define new commands:
\begin{verbatim}
\renewcommand{\P}{\mathbb{P}} 
\newcommand{\as}{\overset{\text{a.s.}}{\longrightarrow}}
\end{verbatim}
and this simplifies the syntax for the definition of almost sure convergence:

\begin{verbatim} $X_n \as X$  if $\P(\omega : (\lim_{n \to \infty} X_n = X) = 1$\end{verbatim}

Use `paragraph' to highlight the start of a new paragraph without defining a new section.

\paragraph{Example highlighted paragraph.}

Paragraphs come in handy when defining multiple variables and highlighting data sources.\footnote{And footnotes come in handy when providing more detail.} By construction, highlighted paragraphs are not indented.\footnote{A second footnote!}

\section{Debugging/Editing Tools}
% !TeX root = ../paper.tex


The preamble contains a few editing tools. Here are some of the most important ones. If you want to see the others, check out the preamble. Many of these tools disappear automatically when you add \texttt{final} to the \Verb"\documentclass" (i.e., \Verb"\documentclass[final]{article}").

\subsection{To-do notes}
Perhaps the command I use the most comes from the \Verb"\todo" command, which I slightly rewrote. I use it to know what else to work on next. For example, I want to remind myself to improve the explanation of the to-do command. What I would do is leave a note like the following:
\forAll{Finalize explanation of to-do.}{Explain how to change colors, how to create additional authors, etc.}


Notice that a hyperlink is created on the "Todo list", and you can go to this list using the up arrow. It is automatically updated once you compile. To create this note, write: 
\begin{verbatim}
\forAll{caption}{summary}
\end{verbatim}

My coauthors also like to use it to leave notes and assign tasks to each other. I defined a couple of new commands (\Verb"\forA" and \Verb"\forB" in the preamble). Imagine we want to make a note so that A knows what to revise:

\forA{Please check that you agree with my explanation above.}{Caption}

Similarly, if instead it was B the one who needed to check something out:
\forB{Check results}{Check this out.}

\subsection{The \xx~command}
Another command I typically use is a placeholder \Verb"\xx" as defined in the preamble. Whenever invoked, it transforms into \xx. This command is useful when writing paragraphs that require much detail. It makes it easy to come back and remove when editing. By construction, \Verb"\xx" leaves no space after itself. You can write \Verb"\xx~" if you want space.



\subsection{The Backreference Command}

The \Verb"\backreference{}" command helps identify where a particular figure, table, or appendix is being cited. For example, citing Table \ref{tab:sample_table1}  and \ref{sec_app:ex} here would mark both of these as cited in this page. To use this command:
\begin{enumerate}
\item Define a label for the figure/table below your caption, or below the section definition with the \Verb"\label{}" command (e.g.,  \Verb"\label{fig:sample_fig}").
\item Invoke the \Verb"\backreference{}" command with the same label (e.g., using the same label as above, \Verb"\backreference{fig:sample_fig}"). You can invoke the command anywhere, but I recommend placing it in the notes section of a figure or a table, or below the section defnition. 
\end{enumerate}




Note that the command requires you to manually add the correct \Verb"\label{}". This backreference can be invoked anywhere. To use it, write \Verb"\backreference{label}".





\section{Mathematical Tools}
% !TeX root = ../paper.tex

We also define and demonstrate how we use the theorem, lemma, definition, and proof environments. Each of them has its own numbering, but that can be redefined (if needed) in the preamble definitions. 


\begin{theorem}[Fermat's Last Theorem]
For integer $n > 2$, no three positive integers $a, b, c$ exist that satisfy $a^n + b^n = c^n$.
\end{theorem}

\begin{proof}
The proof is left to the reader.
\end{proof}

\begin{lemma}[Fatou's Lemma] Given a measure space $(X, \Sigma, \mu)$, let $\{f_n\}$ be a sequence of functions such that $f_n: X \to [0, \infty]$ is measurable for each $n$. Then,
$$
\int_X \bigg(\lim_{n \to \infty} \inf f_n\bigg ) d\mu  \leq \lim_{n \to \infty} \inf \int_X f_n d\mu. 
$$
\end{lemma}

\begin{definition}[Topology]
A topology on a set $X$ is a collection of subsets $\tau$ of $X$ satisfying the following three properties:
\begin{enumerate}
\item $\varnothing \in \tau$ and $X \in \tau$.
\item Closed under finite intersections: If $U_i \in \tau$, $i=1, \ldots, n$, then $U_1 \cap \cdots \cap U_n \in \tau$.
\item Closed under unions: If $\{U_i\}_{i \in I}$ is a collection of elements of $\tau$, then $\bigcup_{i \in I} U_i \in \tau$. 
\end{enumerate}
\end{definition}




% REFERENCES ======================================================
\newpage
\singlespacing
\bibliographystyle{chicago}
\bibliography{library}
%\printbibliography

\newpage
\newgeometry{
	left=1.5cm,
	right=1.5cm,
	top=2.5cm,
	bottom=2.5cm}
%\newgeometry{margin=1.5cm}

% TABLES ========================================================
\section*{Tables}
% !TeX root = ../paper.tex

Table \ref{tab:sample_table1} is an example of a \LaTeX table using the settings in this template. Note the following:

\begin{enumerate}
\item The column width is set using \Verb"\begin{tabular}{L{3cm}C{1.1cm} ... }" where one can use \texttt{L}, \texttt{C}, or \texttt{R} followed by the length \Verb"{XXcm}" using any length measure (\Verb"{cm}", \Verb"{in}", \Verb"{ex}",  etc.). 
\item Alternatively, columns can be defined in the standard way (using \texttt{l}, \texttt{c}, or \texttt{r}).
\item Column (3) is not shown, but exists. This is done by using \texttt{H} in the column width as shown in the code.
\item The max width of the table is set by using \Verb"\begin{adjustbox}{width=\textwidth}".
\item \Verb"\caption{}", which sets the name of the figure in the paper should always precede \Verb"\label{}".
\end{enumerate}


\begin{table}[H]
\caption{Example Table 1}
\label{tab:sample_table1}
\begin{center}
\begin{adjustbox}{width=\textwidth}
	\begin{tabular}{L{3cm}C{1.1cm}C{1.1cm}HC{1.1cm}C{1.1cm}C{1.1cm}C{1.1cm}C{1.1cm}C{1.1cm}C{1.1cm}C{1.1cm}}
\toprule
            &        2010&        2011&        2012&        2013&        2014&        2015&        2016&        2017&        2018&        2019&        2020\\
            &         (1)&         (2)&         (3)&         (4)&         (5)&         (6)&         (7)&         (8)&         (9)&         (10)&         (11)\\
\midrule
\multicolumn{12}{l}{\textit{Panel A. Data Subset 1}} \bigstrut \\
$~$Outcome X &       000 &       000&     000&       000&      000&       000 &       000&     000&       000&      000&      000\\
$~$Outcome Y &       000 &       000&     000&       000&      000&       000 &       000&     000&       000&      000&      000\\
$~$Outcome Z &       000 &       000&     000&       000&      000&       000 &       000&     000&       000&      000&      000\\
\midrule
\multicolumn{12}{l}{\textit{Panel B. Data Subset 2}} \bigstrut \\
$~$Outcome X &       000 &       000&     000&       000&      000&       000 &       000&     000&       000&      000&      000\\
$~$Outcome Y &       000 &       000&     000&       000&      000&       000 &       000&     000&       000&      000&      000\\
$~$Outcome Z &       000 &       000&     000&       000&      000&       000 &       000&     000&       000&      000&      000\\
\bottomrule
\end{tabular}
\end{adjustbox}
\end{center}

{\footnotesize\textit{Notes:} This is an example of a LaTeX table.
\backreference{tab:sample_table1} \par}

\end{table}


%
% FIGURES =======================================================
\newpage
\section*{Figures}
% !TeX root = ../paper.tex

We will look at two types of figure examples: first, where the figure consists of a single image/graph/chart (as in Figure \ref{fig:sample_fig}), and second where the figure consists of multiple subfigures (Figure \ref{fig:sample_subfig}). Some notes:

\begin{enumerate}
\item \Verb"\caption{}", which sets the name of the figure in the paper should always precede \Verb"\label{}".
\item \Verb"\includegraphics{}" will look for the figures in the \Verb"\graphicspath{}" in the preamble.
\end{enumerate}


\begin{figure}[bh!tp]
\caption{An Example Figure}
\label{fig:sample_fig}
\vspace{-1em}
\begin{center}
	\includegraphics[width=0.5\textwidth]{example-image-golden}
\end{center}

{\footnotesize\textit{Notes:} This is an example of a figure added to a \LaTeX~document. 
\backreference{fig:sample_fig} \par}

\end{figure}


\begin{figure}[htbp!]
\caption{An Example Figure with Subfigures}
\label{fig:sample_subfig}
\vspace{-1em}
\begin{center}
\begin{subfigure}{0.49\textwidth}
\caption{Subfigure A}
\centering
\includegraphics[width=\textwidth]{example-image-golden}
\end{subfigure}
\begin{subfigure}{0.49\textwidth}
\caption{Subfigure B}
\centering
\includegraphics[width=\textwidth]{example-image-golden}

\end{subfigure}
\end{center}
{\footnotesize \textit{Notes:} 
This is an example of a \LaTeX~figure. 
Panel (a) displays a large letter A. Panel (b) displays a large letter B.
 \backreference{fig:sample_subfig}
\par}
\end{figure}


% APPENDIX ======================================================
\newpage
\appendix
\newgeometry{margin=2.5cm}
%!TEX root = ../paper.tex

\clearpage
\renewcommand\thefigure{OA-\arabic{figure}}
\renewcommand\thetable{OA-\arabic{table}}
\renewcommand*{\thepage}{OA - \arabic{page}}
\renewcommand\thesection{Appendix \Alph{section}}
\renewcommand\thesubsection{\Alph{section}.\arabic{subsection}}
\setcounter{figure}{0}
\setcounter{table}{0}
\setcounter{page}{1}

\begin{center}
	{\huge\textbf{Paper Title}}\\[1em]
	{\huge\textbf{Paper Subtitle}}\\[1em]
	\large Author 1 and Author 2 \\[1em]
	\Large Appendix - For Online Publication \\[1em]
\end{center}

% TABLE OF CONTENTS BUT ONLY FOR THE SECTIONS IN THE APPENDIX
\renewcommand\cftsecdotsep{\cftdotsep}
\renewcommand\cftsubsecdotsep{\cftnodots}
\renewcommand{\cftsecnumwidth}{6em}
\renewcommand{\cftpnumalign}{r}
%\renewcommand{\cftsecleader}{\normalfont\cftdotfill{\cftsecdotsep}}


\renewcommand{\cftsecleader}{\cftdotfill{\cftsecdotsep}\hspace{1.8em}}


%\renewcommand{\cftsecnumwidth}{6em}
\etocdepthtag.toc{appendix}
\etocsettagdepth{main}{none}
\etocsettagdepth{appendix}{subsection}

\tableofcontents
\newpage


\newpage
\doublespacing
\newgeometry{
	left=1.5cm,
	right=1.5cm,
	top=2.5cm,
	bottom=2.5cm}

\section{Exhibits}
\label{sec_app:ex}
\backreference{sec_app:ex}%
Pages in the Appendix are also labeled with an OA- prefix. Similarly, exhibits in the Appendix are labeled with an OA- prefix. When cited, this prefix is automatically added to the name (e.g., Figure \ref{fig:sample_fig_app}).

\begin{figure}[H]
\caption{An Example Figure in the Appendix}
\label{fig:sample_fig_app}
\vspace{-1em}
\begin{center}
	\includegraphics[width=0.5\textwidth]{example-image-golden}
\end{center}

{\footnotesize\textit{Notes:} This is an example of a figure added to a \LaTeX \ document. 
\backreference{fig:sample_fig_app} \par}



\end{figure}




\end{document}