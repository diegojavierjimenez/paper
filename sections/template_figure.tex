% !TeX root = ../paper.tex

We will look at two types of figure examples: first, where the figure consists of a single image/graph/chart (as in Figure \ref{fig:sample_fig}), and second where the figure consists of multiple subfigures (Figure \ref{fig:sample_subfig}). Some notes:

\begin{enumerate}
\item \Verb"\caption{}", which sets the name of the figure in the paper should always precede \Verb"\label{}".
\item \Verb"\includegraphics{}" will look for the figures in the \Verb"\graphicspath{}" in the preamble.
\end{enumerate}


\begin{figure}[bh!tp]
\caption{An Example Figure}
\label{fig:sample_fig}
\vspace{-1em}
\begin{center}
	\includegraphics[width=0.5\textwidth]{example-image-golden}
\end{center}

{\footnotesize\textit{Notes:} This is an example of a figure added to a \LaTeX~document. 
\backreference{fig:sample_fig} \par}

\end{figure}


\begin{figure}[htbp!]
\caption{An Example Figure with Subfigures}
\label{fig:sample_subfig}
\vspace{-1em}
\begin{center}
\begin{subfigure}{0.49\textwidth}
\caption{Subfigure A}
\centering
\includegraphics[width=\textwidth]{example-image-golden}
\end{subfigure}
\begin{subfigure}{0.49\textwidth}
\caption{Subfigure B}
\centering
\includegraphics[width=\textwidth]{example-image-golden}

\end{subfigure}
\end{center}
{\footnotesize \textit{Notes:} 
This is an example of a \LaTeX~figure. 
Panel (a) displays a large letter A. Panel (b) displays a large letter B.
 \backreference{fig:sample_subfig}
\par}
\end{figure}
