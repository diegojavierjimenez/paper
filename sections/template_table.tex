% !TeX root = ../paper.tex

Table \ref{tab:sample_table1} is an example of a \LaTeX table using the settings in this template. Note the following:

\begin{enumerate}
\item The column width is set using \Verb"\begin{tabular}{L{3cm}C{1.1cm} ... }" where one can use \texttt{L}, \texttt{C}, or \texttt{R} followed by the length \Verb"{XXcm}" using any length measure (\Verb"{cm}", \Verb"{in}", \Verb"{ex}",  etc.). 
\item Alternatively, columns can be defined in the standard way (using \texttt{l}, \texttt{c}, or \texttt{r}).
\item Column (3) is not shown, but exists. This is done by using \texttt{H} in the column width as shown in the code.
\item The max width of the table is set by using \Verb"\begin{adjustbox}{width=\textwidth}".
\item \Verb"\caption{}", which sets the name of the figure in the paper should always precede \Verb"\label{}".
\end{enumerate}


\begin{table}[H]
\caption{Example Table 1}
\label{tab:sample_table1}
\begin{center}
\begin{adjustbox}{width=\textwidth}
	\begin{tabular}{L{3cm}C{1.1cm}C{1.1cm}HC{1.1cm}C{1.1cm}C{1.1cm}C{1.1cm}C{1.1cm}C{1.1cm}C{1.1cm}C{1.1cm}}
\toprule
            &        2010&        2011&        2012&        2013&        2014&        2015&        2016&        2017&        2018&        2019&        2020\\
            &         (1)&         (2)&         (3)&         (4)&         (5)&         (6)&         (7)&         (8)&         (9)&         (10)&         (11)\\
\midrule
\multicolumn{12}{l}{\textit{Panel A. Data Subset 1}} \bigstrut \\
$~$Outcome X &       000 &       000&     000&       000&      000&       000 &       000&     000&       000&      000&      000\\
$~$Outcome Y &       000 &       000&     000&       000&      000&       000 &       000&     000&       000&      000&      000\\
$~$Outcome Z &       000 &       000&     000&       000&      000&       000 &       000&     000&       000&      000&      000\\
\midrule
\multicolumn{12}{l}{\textit{Panel B. Data Subset 2}} \bigstrut \\
$~$Outcome X &       000 &       000&     000&       000&      000&       000 &       000&     000&       000&      000&      000\\
$~$Outcome Y &       000 &       000&     000&       000&      000&       000 &       000&     000&       000&      000&      000\\
$~$Outcome Z &       000 &       000&     000&       000&      000&       000 &       000&     000&       000&      000&      000\\
\bottomrule
\end{tabular}
\end{adjustbox}
\end{center}

{\footnotesize\textit{Notes:} This is an example of a LaTeX table.
\backreference{tab:sample_table1} \par}

\end{table}

